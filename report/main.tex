\documentclass[11pt, letterpaper]{article}  % change to >11 pt if you like, and change article with report
\usepackage[letterpaper, top=3.71cm, bottom=3.20cm, left=2.86cm, right=2.86cm]{geometry}
\usepackage[utf8]{inputenc}
\usepackage{natbib}
\usepackage{graphicx}
\usepackage{color}
\usepackage{subfig}
\usepackage{float}
\usepackage{hyperref}
\usepackage{url}
%\usepackage{pythontex}
\definecolor{bg}{gray}{0.95}
\usepackage{minted}
\usepackage{wrapfig}

\title{\vspace{-2cm}\textbf{Data Mining project report}}
\author{\textbf{\small{\textit{Dalla Noce Niko, Ristori Alessandro, Lombardi Giuseppe}}} \\ % put your full name here
        \small{Master Degree in Computer science.}\\ \small{{n.dallanoce@studenti.unipi.it, a.ristori5@studenti.unipi.it, g.lombardi11@studenti.unipi.it}.} \\  % put your Master Degree here
        \small{Data Mining, Academic Year: 2021/2022} \\
        \small{Date: 5/01/2022} \\
       \textbf{\small{\url{https://github.com/nikodallanoce/DataMiningProject}}}
}

\renewcommand\refname{} %remove this line to automatically show the bibliography header

\begin{document}

\nocite{*}  % comment this line to list only the articles you really cite
\date{}
\maketitle
\begin{center}
    \includegraphics[width=0.2\textwidth]{images/unipi.png}\\
    \vspace{0.5cm}
\end{center}
%\begin{abstract}
%The report starts with a long, but needed introduction to the main concepts that we've seen during the course of our work and it shows what was our purpose and how we reached it. Then it shows the results obtained by the models we tried giving at the same time the reasons of their performances. The report concludes with our final considerations on what and how we could have done more.
%\end{abstract}
\newpage
\tableofcontents
\newpage
%\listoffigures
%\include{chapters/introduction}
\section{Data understanding and preparation}
We focused our work on the matches dataset that contains tennis matches played in many tournaments across the world over the last five years, it includes both male and female players and has 186.129 records and 49 features. We decided to not do any cleaning or integration on the male and female datasets since they were only used to get the sex for each player.

\subsection{Data understanding}
The data understanding task focuses on analyzing the dataset and its integration by removing duplicated values, fixing missing values and solving the possible outliers.
\subsubsection{Data semantics}
In this phase we focused on understanding the meaning of each feature inside the dataset.
\begin{itemize}
    \item \textbf{tourney\_id}: it's a \textbf{string} that uniquely identifies each tournament. It is mainly composed by two part separated by a dash "-", where the first one indicates the year when the tournament was played and the second one refers to the tournament identifier, i.e. '2018-W-WITF-EGY-03A-2018'. The dataset contains some null value for this feature (only for the last tournament), while it has 4853 unique values.
    \item \textbf{tourney\_name}: it's a string that represent the name of the tournament and it's not unique because the name of the tournament usually is the same over the years. It has some missing value only in the last part of the dataset.
    \item \textbf{surface}: it indicates the surface where the matches took place. It's a categorical attribute which its possible values are: hard, clay, grass and carpet. Only few matches are missing this feature.
    \item \textbf{draw\_size}: it's a float that indicates the number of player in a tournament.
    \item \textbf{tourney\_level}: it's a string indicates whether the tournament is for male or female players and also its level (mastr, ATP500, ATP1000 etc.). Has some missing values in the last part of the dataset just like for the tourney\_id.
    \item \textbf{tourney\_date}: it's the date when the match took place. In the dataset it's a float in the form YYYYMMDD. To use it as real date, a parsing to a date type is needed. Missing values are presents only in the last part of the dataset.
    \item \textbf{match\_number}: it's a float number that represents the match number in a tournament. Due to its meaning, it should be parsed to an integer.
    \item \textbf{winner\_id}: in the dataset, it's a float representing the id of the match winner. For sure, it is unique for the player with the same gender, while the same id can appear for a male player and for a female player as well, since they play in different tours (ATP and WTA).
    \item \textbf{winner\_entry}: it's a string that indicates how a player joined the tournament. There are a lot of missing values for this feature.
    \item \textbf{winner\_hand}: it's a string representing the winner player's favourite hand. Its possible values are 'U' for unknown, 'R' for right and 'L' for left. Actually there are also null values, but they can be treated as unknown.
    \item \textbf{winner\_ht and loser\_ht}: float representing the winner player's height. There are a lot of missing values for this field that can not be inferred by other records where this it's filled.
    \item \textbf{winner\_ioc and loser\_ioc}: a 3-character string representing the player's nationality. Some of them have been written using different standards. For instance, a German player is present both as 'GER' and as 'DEU'. Also, some players appears with more than one nationality because they switched nationality during the last years.
    \item \textbf{winner\_age and loser\_age}: a float that indicates the age of the winner. The decimal part represents the percentage of the days days left to birthday. Some of this values are missing in the dataset, but few of them can be precisely inferred.
    \item \textbf{best\_of}: a float indicating the number of set for a match. We parse it to an integer.
    \item \textbf{round}: the match round (e.g. F stands for final and SF for semifinal).
    \item \textbf{minutes}: indicates how much a match lasted. It's a float that we converted to an integer.
    \item \textbf{w\_ace and l\_ace}: number of aces (valid serve won, first or second, not touched by the opposing player). It's a float, but actually is an integer.
    \item \textbf{w\_df and l\_df}: number of double faults committed by the player (more specifically the number of invalid second serves). It's a float converted to int.
    \item \textbf{w\_svpt and l\_svpt}: this may be confusing, it isn't the number of points obtained after a player serve but the number of serves done by the player (the former is referred on specialised sites as serve points won).
    \item \textbf{w\_1stIn and l\_2stIn}: number of valid first serve by the player.
    \item \textbf{w\_1stWon and l\_1stWon}: number of points won after a valid first serve by the player.
    \item \textbf{w\_2ndWon and l\_2ndWon}: number of points won after a valid second serve by the player, be aware that a point lost after a second serve could be a double fault (this means that the second serve wasn't valid).
    \item \textbf{w\_SvGms and l\_SvGms}: number of games (not points) where the player was serving.
    \item \textbf{w\_bpSaved and l\_bpSaved}: number of breakpoints (the player is a point from losing a game where he's serving) saved (the player serving won the point).
    \item \textbf{w\_bpFaced and l\_bpFaced}: number of breakpoints that the player faced (see previous feature).
    \item \textbf{w\_rank and l\_rank}: the rank of the player, in tennis the points are awarded based on the performance at each tournament.
    \item \textbf{w\_rank\_points and l\_rank\_points}: the ranking points of the player.
    \item \textbf{tourney\_revenue}: the tourney revenue coming from ticket sales etc. it isn't the prize of the tournament (which would have been more interesting for a player analysis).
    \item \textbf{tourney\_spectators}: number of spectators of the entire tournament.
\end{itemize}

\subsubsection{Type casting}
In this dataset, most of the features have an incorrect data type. We parse the tourney\_date from float to pandas date type. Furthermore, every float attribute has been parsed to integer except for the winner/loser age and the tourney\_revenue.

\subsubsection{Dropping useless matches}
There are many matches without any statistics, highlighted in \autoref{fig:features_heatmap}. Such matches are from minor tournaments and the players who played in them don't have many  matches in the dataset, we decided to drop them.
\begin{figure}[H]
    \centering
    \includegraphics[width= 0.55\linewidth]{images/data_understanding/features_heatmap.png}
    \caption{The missing values heatmap.}
    \label{fig:features_heatmap}
\end{figure}
Furthermore, starting from line n. 186073 to the end, we found a block of records that is a copy of some matches of the Taipei tournament, but with some field left empty. Having noticed this, we dropped this useless part of the dataset.

\subsubsection{Dropping useless features:}
For our purpose, an analysis of the players, some features are not interesting or straight useless, so we decided to drop them. We don't need the \textbf{tourney name}, since we don't differentiate between the tournaments (masters atp, ATP1000 etc.), moreover we already have the \textbf{tourney id} and \textbf{date}.

The \textbf{draw size} is useless for the players and the \textbf{tourney level} can't be used to differentiate between men and women, since the latter can play the same kind of tournaments of the former. For what concerns \textbf{match num}, it's a progressive number for the matches of a tournament, sometimes is totally arbitrary, we don't need it.

\textbf{Winner entry} (and \textbf{loser entry}) is a string that shows if the player qualified by a wildcard, as a lucky loser etc., it's not a very useful attribute since no site lists them, moreover is the feature with the most missing values.

The last features we dropped are \textbf{tourney revenue} and \textbf{tourney spectators}, the former is the amount of revenue generated by the tournament and not the winner prize which would have been really interesting. The latter is the amount of spectators of the entire tournament, not match to match, so we can't retrieve information about how famous the players involved in a certain match are or if they can play better under more pressure.

\subsubsection{Dropping duplicates}
We found that the matches dataset has 302 duplicated records. Due to the domain of this dataset, these duplicates are useless, so we drop them.

\subsubsection{Data integration}
The data integration task consists in solving the various issues found in the dataset by filling the missing values, where it's possible, and fixing the various problems that may arise (e.g.: players with two hands preferred or with more than one ioc).
\paragraph{Player id and name}
In our analysis, we found that there are three players ('Guy Stokman', 'Giuseppe Tresca', 'Kuan Yi Lee') that appear in the dataset with more than one id associated. Moreover, there's only one player with an actual homonym (Kuan Yi Lee), one is a male (id: 134120) and the other one is female (id: 221745), but we discovered that the actual name of the male player is Kuan-Yi, so we change his name to differentiate between the two players. For what concerns the other two players, we assign them their last id.

Deepening the analysis, we found that ids are shared only between a male and female player and this is allowed since there exist different tours (ATP and WTA) for each sex, so we don't need to apply any change, moreover we won't work with ids so this won't affect our analysis.

\paragraph{Surface}
We found that the "surface" feature has some null values. Moreover, the number of tournaments without a surface are 42 while the number of matches without a surface are 117. In order to solve this problem, for those tournaments that lack the surfac  we can retrieve it if at least one match of the same tournament has it. Unluckily there are no tournaments for which we can retrieve the surface (all of them are from the Davis or Federation cup, where the surface changes from event to event), so we chose to sample such values from the distribution of the surfaces.

\paragraph{Winner/loser hand}
We have to manage the missing values for "winner\_hand" and "loser\_hand". To do this, we checked if the value is missing only for a particular player or if he/she never has the hand defined and, in this case, we assign it to him/her. After this step, however, the missing values are still present, but since the domain of this feature is {'U', 'L', 'R'}, we can safely treat null values as 'U'. The players with unknown hand are 1840. Among them, we retrieved the actual preferred hand for one player, Amina Anshba who is left-handed.

\paragraph{Winner/loser height}
This feature contains missing values too, so we address this problem as before, looking for those players that have a known height somewhere in the dataset. A deeper analysis show us that 'David Goffin' has two heights registered: 180 cm and 163 cm. Since his real height is the first one, we fixed it in in all occurrences in which he appears.

\paragraph{Winner/loser ioc}
Since there are no missing values for this feature, we only need to check if there are players with more than one ioc. Since we dropped records and, as a consequence, players that have played too few matches, we haven't found players with more than one ioc. In the original dataset instead, it happens. Anyway, we found that player called 'Xinmu Zhou', has 'UNK' nationality, but he is Chinese actually, so we assign him 'CHN'. Lastly, we discovered that sometimes, the acronyms of the nations are wrong, since some of them don't follow the international standard (e.g.: GRE for Greece instead of GRC, or TPE for Taiwan instead of TWN) or have both at the same time (e.g.: DEU and GER for Germany), so we fixed them, at least the ones we found.

\paragraph{Winner/loser age}
There are 58 players with unknown age. Among them there are only two player which age can be inferred from the other information inside the dataset, we fiex the missing ages during the data preparation task. 

\paragraph{Winner/loser rank} This information is missing for those players that played only few matches. We fixed the missing values with the last tournament rank for each player (or the next if not present) that already had a rank. For those without rank we assumed they didn't have any points so we assigned them the maximum rank in found in the dataset.

\paragraph{Winner/loser rank points} Same work as for the rank feature, instead this time we assign the minimum value for those players without any rank points (they are the same we found during the rank integration).

\subsubsection{Outlier detection}
Given that all the feature distribution are Gaussian or half normal, to analyze outliers we must first compute the first quartile, the third quartile, the median and the interquartile for each feature. Then we compute the lower bound $L=Q1 - 1.5 * IQR$ and the upper bound $U=Q1 + 1.5 * IQR$ for each feature. In case some numerical data was less or greater than the lower or upper bound respectively, we can identify an outlier. For half-normal distribution we don't consider the lower bound. In case of an outlier that can be fixed with a real value, we do it. In this way we found, for example, two players that have too low height and, in fact, these data were wrong. We fixed it with the players' real height. Whereas high values correspond to those players that are actually tall. We have applied this analysis for every numerical feature.
\begin{figure}[H]
    \centering
    \subfloat[Before detection]{{\includegraphics[width= 0.5\linewidth]{images/data_understanding/ht_before_detection.png}}}%
    \subfloat[After detection]{{\includegraphics[width= 0.5\linewidth]{images/data_understanding/ht_after_detection.png}}}%
    \caption{An example of a feature's distribution after dealing with outliers.}
    \label{fig:before_and_after_detection}
\end{figure}

\subsubsection{Correlations}
We plot thr correlation matrix in order to visualize whether there are correlation between the features. In \autoref{fig:corr_matrix_tennis_matches}, the more the color is red the more the features are correlated.
\begin{figure}[H]
    \centering
    \includegraphics[width= 0.74\linewidth]{images/data_understanding/correlation.png}
    \caption{Correlation matrix of the tennis matches dataset.}
    \label{fig:corr_matrix_tennis_matches}
\end{figure}
After looking at the correlation matrix we dropped the minutes feature since we deemed it as useless after doing a huge help during the outlier detection by showing "inusual matches". 
\subsection{Data preparation}
In this section we focus on describing our new dataset, cleaned and composed by some new features derived by the original ones.

\subsubsection{Building the player's profile}
The purpose of data preparation is building a profile that will be feeded to the clustering analysis, as a starting point we decided to build such profile from the features we had from the matches dataset.

\paragraph{Sex}
First of all, when we started building our new dataset, we assigned the sex to each player inside tennis\_matches using the female\_players and male\_players datasets. Some players of the tennis\_matches dataset were not found in the "sex" dataset due to spelling errors in their name, but we solved this problem by looking online.
\begin{figure}[H]
    \centering
    \subfloat[Sex distribution]{\includegraphics[width=0.20\linewidth]{images/data_preparation/sex_distribution.png}}
    \subfloat[Age distribution]{\includegraphics[width=0.32\linewidth]{images/data_preparation/age_distribution.png}}
    \caption{Sex and age distribution}
    \label{fig:sex_age_feature}
\end{figure}
\paragraph{Age}
Since we are doing a "current" analysis, we assign to each one of the players the last age they appear in the original dataset. Some of them have an unknown one, we had to sample it.

\paragraph{Ioc} This was the easiest one since we didn't have to deal with missing values.
\begin{figure}[H]
    \centering
    \includegraphics[width=0.74\linewidth]{images/data_preparation/ioc_distribution.png}
    \caption{IOC distribution}
    \label{fig:ioc_feature}
\end{figure}

\paragraph{Height} We solve the issue of having many players without an height, for this purpose we sample their height from the distributions based on countries and sex where this is possible, otherwise we assign them the mean height of their respective sex.

\paragraph{Hand} Just like for the ioc we just needed to assign the hand to the players since the missing values were already dealt with.

\paragraph{Wins and losses} We calculated the total matches won or lost by the player and also how many matches they won on a specific surface.

\paragraph{Tournaments won} We calculated the tournaments won by the player by looking at how many times he appeared in a final as the winner.

\paragraph{Surfaces} We inserted the amount of wins by a player for each surface he/she played on.

\paragraph{Statistics, rank and rank points} For each player we calculated all the statistics coming from the cleaned matches dataset, futhermore we assigned at each player their rank and rank points, eliminating from the dataframe those played less than ten matches.

\subsubsection{Building new features}
Having inserted the feature coming from the cleaned matches dataset it was time to look at the correlation between the features.
\begin{figure}[H]
    \centering
    \includegraphics[width= 0.59\linewidth]{images/data_preparation/correlations_before_new_features.png}
    \caption{Correlation matrix of the new dataset with the features coming from the matches dataset.}
    \label{fig:corr_matrix_old_features}
\end{figure}
As we can see from \autoref{fig:corr_matrix_old_features}, there were many highly-correlated feature, so we decided to build new ones from those that we had in the dataframe, such as [\textit{num\_matches, p\_wins, p\_w\_Hard, p\_w\_Clay, p\_w\_Grass, mean\_ace and p\_aces, mean\_double\_faults and p\_double\_faults, mean\_1st\_in and p\_1st\_in, mean\_1st\_won and p\_1st\_won, mean\_2nd\_won and p\_2nd\_won,\\mean\_bp\_saved and p\_bp\_saved, mean\_bp\_faced, mean\_sv\_games, mean\_sv\_points}]

\paragraph{Categorical features} We build categorical attributes that split players by age, height and rank ranges.
\vspace{3mm}

We then dropped the features that were too high correlated with another one or we deemed useless for the players analysis (n\_matches, mean\_aces, mean\_sv\_games, mean\_double\_faults).
After adding the new derived features, we checked that they were not correlated each other by calculating again the correlation matrix, as showed in \autoref{fig:corr_matrix_new_ds}.

\begin{figure}[H]
    \centering
    \includegraphics[width=0.7\linewidth]{images/data_preparation/players_corr_features.png}
    \caption{Correlation matrix of the new dataset with the feature we added.}
    \label{fig:corr_matrix_new_ds}
\end{figure}
\section{Clustering analysis}
\subsection{K-Means}
\subsection{DBSCAN}
\subsection{Hierachical Clustering}
\section{Predictive Analysis}
\subsection{Data Preparation}
Our goal is to classify the players in two categories: 'strong' players and the 'weak' players.
To do this, we started from the dataframe of the players used in the clustering phase and we labeled them in the two aforementioned classes, exploiting the rank of each player. We decided that the players whose rank fall in a category that is top 5, top 10, top 25, top 50 or top 100 are labelled as strong, thus, all the others as 'weak'. For the male players, we also label as 'strong' the one whose rank is 'top 250'.

Furthermore, we removed some features that we considered irrelevant, such as the players' hand obtained previously through a sampling process and the win rates on different surfaces, but keeping the average number of wins for each surface. Furthermore, we also didn't care about the players' nationality because it's useless for the classification purpose (a player's nationality can't determine his actual strength).

Finally, we removed the rank range, ranking points and ranking, to ensure that the classifier doesn't learn our labeling strategy.

\textit{The classification was done by dividing the male players from the female ones}, this means that there are some models trained using the 'male' data set that work only with the male players data and other ones that work with the female one.
Initially, we conducted the classification analysis by using our entire dataset which is made by both male and female players, the latter are present in minority, about 32\% of the entire data set. By doing in this way the classification metrics were good, but this happened due to the fact that our data set is imbalanced with respect to the players' sex. In fact, we tried to build two test sets based on the players' sex and we discovered that the classification models worked well on the 'male' set, while their performance dropped down on the 'female' set. For this reason, we present our analysis for this section by keeping male and female players separated.

\subsection{Imbalanced data}
Our labeling strategy carried out an imbalanced data set, which is likely to happen in competitive sports. The problem is that the classifiers trained on imbalanced data sets can perform bad as they tend to overfit towards the majority class. In our case, the male dataset has a distribution of the 'strong' and 'weak' players which is circa 74\% and 26\% respectively, totalling 1102 players. The female data set is also imbalanced to the 'weak' class, having a 76/24 percentage distribution, totalling 498 players. To overcome this problems we exploited the SMOTE approach for over-sampling the minority class combined with an under-sampling on the majority class. In this way the distribution of the players is 55\% for the 'weak' class and 45\% for the 'strong' one.

\subsection{Classification methods}
Before applying any of the following classification algorithms, we splitted the dataset into development and test sets, the latter consisting of 10\% of the original data set. Furthermore we used the StandardScaler algorithm from the scikit-learn library to standardize our data.

The following results have been obtained by exploiting the best parameters carried out by the K-Fold cross validation, with K equal to 4, comparing the results by measuring the accuracy score between the K folds. For the lack of space, we will only show the graphical results for the male players, whereas in \autoref{subsect:comparison} we study the performance obtained by our models in both the male and female data set.

\paragraph{Evaluation} In order to evaluate the goodness of our models, we'll show classification metrics like accuracy, precision, recall and F1 score. The miss-classified patterns can be numerically viewed by using the confusion matrix (label 0 for 'weak', 1 for 'strong') and graphically by plotting the ROC curve. Finally, we'll show the "global" weight that each feature had in the classification of the tennis players for those models that offer this capability. Talking about the test set, it is made up of imbalanced data, having the same distribution as the original dataset, thus it's without any sampling strategy applied. This means that any of the following algorithm must beat the dummy classifier.


\subsubsection{Decision tree}
Decision tree is a classification method which produces interpretable results. It's not one of the best algorithms we tested, but, anyway, its performances are pretty good.
\begin{figure}[H]
    \centering
    \subfloat[Confusion matrix]{\includegraphics[width=0.28\linewidth]{images/predictive_analysis/dec_tree/tree_conf_m.png}}
    \subfloat[Features importance]{\includegraphics[width=0.35\linewidth]{images/predictive_analysis/dec_tree/tree_imp_m.png}}
    \subfloat[ROC curve]{\includegraphics[width=0.35\linewidth]{images/predictive_analysis/dec_tree/tree_roc_m.png}}\\
    \subfloat[Decion tree splitting]{\includegraphics[width=0.8\linewidth]{images/predictive_analysis/dec_tree/tree_plot.png}}
    \caption{Results for the Decision Tree classifier}
    \label{fig:DTResults}
\end{figure}

\subsubsection{Naive Bayes}
The Naive Bayes classifier is a "probabilistic classifier" based on applying Bayes' theorem with strong (naïve) independence assumptions between the features.
\begin{figure}[H]
    \centering
    \subfloat[Confusion matrix]{\includegraphics[width=0.28\linewidth]{images/predictive_analysis/naive_bayes/naive_conf_m.png}}
    \subfloat[ROC curve]{\includegraphics[width=0.35\linewidth]{images/predictive_analysis/naive_bayes/naive_roc_m.png}}
    \caption{Results for the Naive Bayes classifier}
    \label{fig:NBResults}
\end{figure}

\subsubsection{Random forest}
It's an ensemble method specifically designed for decision trees, it combines the predictions made by multiple decision trees and outputs the class that is the mode of the class' output by individual trees.
\begin{figure}[H]
    \centering
    \subfloat[Confusion matrix]{\includegraphics[width=0.28\linewidth]{images/predictive_analysis/rand_forest/rf_conf_m.png}}
    \subfloat[Features importance]{\includegraphics[width=0.35\linewidth]{images/predictive_analysis/rand_forest/rf_imp_m.png}}
    \subfloat[ROC curve]{\includegraphics[width=0.35\linewidth]{images/predictive_analysis/rand_forest/rf_roc_m.png}}
    \caption{Results for the Random Forest classifier}
    \label{fig:RFResults}
\end{figure}

\subsubsection{Adaptive boosting (AdaBoost)}
It can be used as ensembler of based classifiers to improve performance. The output of the base learning algorithms ('weak learners') is combined into a weighted sum that represents the final output of the boosted classifier. AdaBoost is adaptive in the sense that subsequent weak learners are tweaked in favor of those instances misclassified by previous classifiers.
\begin{figure}[H]
    \centering
    \subfloat[Confusion matrix]{\includegraphics[width=0.28\linewidth]{images/predictive_analysis/ada_boost/ada_conf_m.png}}
    \subfloat[Features importance]{\includegraphics[width=0.35\linewidth]{images/predictive_analysis/ada_boost/ada_imp_m.png}}
    \subfloat[ROC curve]{\includegraphics[width=0.35\linewidth]{images/predictive_analysis/ada_boost/ada_roc_m.png}}
    \caption{Results for the AdaBoost classifier}
    \label{fig:AdaBoostResults}
\end{figure}

\subsubsection{Rule based}
The rule-based classifier is a classification scheme that makes use of IF-THEN rules for class prediction. 
\begin{figure}[H]
    \centering
    \subfloat[Confusion matrix]{\includegraphics[width=0.28\linewidth]{images/predictive_analysis/rule_based/rule_conf_m.png}}
    \subfloat[ROC curve]{\includegraphics[width=0.35\linewidth]{images/predictive_analysis/rule_based/rule_roc_m.png}}
    \caption{Results for the Rule Based classifier}
    \label{fig:RBResults}
\end{figure}

\subsubsection{K-nearest neighbors (KNN)}
The KNN is an instance based classifier, it uses class labels of nearest neighbors to determine the class label of unknown records.
\begin{figure}[H]
    \centering
    \subfloat[Confusion matrix]{\includegraphics[width=0.28\linewidth]{images/predictive_analysis/knn/knn_conf_m.png}}
    \subfloat[ROC curve]{\includegraphics[width=0.35\linewidth]{images/predictive_analysis/knn/knn_roc_m.png}}
    \caption{Results for the KNN classifier}
    \label{fig:KNNResults}
\end{figure}

\subsubsection{Support-vector machine (SVM)}
SVM is a robust classifier based on statistical learning frameworks, it maps training examples to points in space to maximise the width of the gap between the two categories. New examples are then mapped into that same space and predicted to belong to a category based on which side of the gap they fall.
\begin{figure}[H]
    \centering
    \subfloat[Confusion matrix]{\includegraphics[width=0.28\linewidth]{images/predictive_analysis/svm/svm_conf_m.png}}
    \subfloat[ROC curve]{\includegraphics[width=0.35\linewidth]{images/predictive_analysis/svm/svm_roc_m.png}}
    \caption{Results for the SVM classifier}
    \label{fig:SVMResults}
\end{figure}

\subsubsection{Neural network}
We developed a FF neural network made up of two hidden layers whose neurons are 'activated' by the elu function and a softmax on the output layer. To ensure the generalization capability, we use the dropout technique and the layer normalization, to stabilize the training phase. In the end, we used the categorical cross entropy as loss function.
\begin{figure}[H]
    \centering
    \subfloat[Confusion matrix]{\includegraphics[width=0.28\linewidth]{images/predictive_analysis/nn/nn_conf_m.png}}
    \subfloat[Learning curve]{\includegraphics[width=0.35\linewidth]{images/predictive_analysis/nn/nn_learnCurve_m.png}}
    \subfloat[ROC curve]{\includegraphics[width=0.35\linewidth]{images/predictive_analysis/nn/nn_roc_m.png}}
    \caption{Results for the Neural Network classifier}
    \label{fig:NNResults}
\end{figure}

\subsubsection{TabNet}
TabNet \cite{arik2020tabnet} is a novel high-performance and interpretable canonical deep tabular data learning architecture. TabNet uses sequential attention to choose which features to reason from at each decision step, enabling interpretability and more efficient learning as the learning capacity is used for the most salient features.
\begin{figure}[H]
    \centering
    \subfloat[Global feature importance]{\includegraphics[width=0.35\linewidth]{images/predictive_analysis/tab_net/tab_imp_m.png}}
    \subfloat[Local feature importance]{\includegraphics[width=0.33\linewidth]{images/predictive_analysis/tab_net/tab_impLoc_m.png}}
    \subfloat[ROC curve]{\includegraphics[width=0.30\linewidth]{images/predictive_analysis/tab_net/tab_roc_m.png}}
    \caption{Results for the Tab Net classifier}
    \label{fig:TNResults}
\end{figure}

\subsection{Comparison} \label{subsect:comparison}
In this section we show how the aforementioned algorithms performed in both the female and male datasets. In \autoref{tab:female_compare} and \autoref{tab:male_compare} we reported the performances in percentages. For each numerical entry, the first number is referred to the 'weak' class and the second one to the 'strong' class. In general, there wasn't a model that performed well in both the two data set as AdaBoost obtained the best performance on the female data set, whereas Random Forest in the male one. In the end, we can assert that AdaBoost was the best model overall, which is a good result because it is also interpretable. For this reason, we also compared the feature importance that this model associates to the male players with respect to the female players, as shown in \autoref{fig:FeatImpSex}.

\begin{table}[H]
\footnotesize
\centering
%\tiny
\begin{tabular}{c|c|c|c|c|c|c|c|c}\hline \hline
\multicolumn{9}{c}{\textbf{Female players classification}}\\ \hline \hline
\multirow{2}{*}{\textbf{Model}} & \multicolumn{4}{c|}{\textbf{Training}} & \multicolumn{4}{c}{\textbf{Test}} \\\cline{2-9}
& Prec. & Rec. & F1 & Acc. & Prec. & Rec. & F1 & Acc. \\\hline
Decision Tree & 95/89 & 91/94 & 93/91 & 92 & 96/62 & 82/89 & 89/73 & 84 \\
Naive Bayes & 91/84 & 86/90 & 89/87 & 88 & 96/62 & 82/89 & 89/73 & 84\\
\rowcolor{brown!50} Random Forest & 100/97 & 98/100 & 99/98 & 99 & 95/75 & 91/83 & 93/79 & 89\\
\rowcolor{yellow!70} AdaBoost & 100/100& 100/100 & 100/100 & 100 & 95/83 & 95/83 & 95/83 & 92\\
Rule Based & 90/93 & 95/87 & 92/90 & 91 & 91/48 & 74/78 & 82/60 & 75\\
KNN & 95/86 & 88/95 & 91/90 & 91 & 94/68 & 88/83 & 91/75 & 87\\
\rowcolor{brown!50} SVM & 100/95 & 96/100 & 98/97 & 97& 95/75 & 91/83 & 93/79 & 89\\
\rowcolor{gray!40} Neural Network & 92/87 & 89/91 & 91/89 & 90 & 98/71 & 88/94 & 93/81 & 89\\
TabNet & 97/85 & 86/97 & 91/91 & 91 & 93/70 & 89/78 & 91/74 & 87\\\hline \hline
\end{tabular}
\caption{A comparison between the performance of the algorithms employed to classify the female players.}
\label{tab:female_compare}
\end{table}

\begin{table}[H]\footnotesize
\centering
%\tiny
\begin{tabular}{c|c|c|c|c|c|c|c|c}\hline \hline
\multicolumn{9}{c}{\textbf{Male players classification}}\\ \hline \hline
\multirow{2}{*}{\textbf{Model}} & \multicolumn{4}{c|}{\textbf{Training}} & \multicolumn{4}{c}{\textbf{Test}} \\\cline{2-9}
& Prec. & Rec. & F1 & Acc. & Prec. & Rec. & F1 & Acc. \\\hline
Decision Tree & 97/95 & 96/96 & 96/96 & 96 & 99/81 & 92/98 & 95/88 & 93 \\
Naive Bayes & 94/88 & 89/92 & 92/90 & 91 & 97/76 & 90/91 & 93/83 & 90\\
\rowcolor{yellow!70} Random Forest & 100/97 & 98/100 & 99/99 & 99 & 99/84 & 93/98 & 96/90 & 95\\
AdaBoost & 95/91 & 93/94 & 94/93 & 93& 99/79 & 91/98& 95/88 & 93\\
Rule Based & 95/93 & 95/94 & 95/94 & 94 & 92/65 & 85/79 & 89/72 & 84\\
KNN & 96/94 & 95/95 & 96/95 & 95 & 92/82 & 93/84 & 94/83 & 91\\
SVM & 100/97 & 98/99 & 99/98 & 98 & 95/88 & 96/86 & 96/87 & 93\\
\rowcolor{gray!40} Neural Network & 93/92 & 94/91 & 93/92 & 93 & 98/85 & 94/95 & 96/90 & 95\\
\rowcolor{brown!50}TabNet & 94/88 & 90/93 & 92/90 & 91 & 97/85 & 94/93 & 96/89 & 94\\\hline \hline
\end{tabular}
\caption{A comparison between the performance of the algorithms employed to classify the male players.}
\label{tab:male_compare}
\end{table}

\begin{figure}[H]
    \centering
    \subfloat[Features importance for female]{\includegraphics[width=0.35\linewidth]{images/predictive_analysis/comparison/ada_featImp_f.png}}
    \subfloat[Features importance for male]{\includegraphics[width=0.35\linewidth]{images/predictive_analysis/ada_boost/ada_imp_m.png}}
    \caption{Features importance to classify a player based on its sex.}
    \label{fig:FeatImpSex}
\end{figure}
\section{Time Series Analysis}
We decided to do task 4.1 as last part of the project which expected us to find groups of similar cities given their temperature trends.
\autoref{fig:timeseries_dataset} shows the initial dataset and the modified one, the latter, built with the pivot method from pandas, was the one we worked on during our analysis. The resulting dataframe now has the cities as indexes and the temperature records as values, so it made our task easier.

\begin{figure}[H]
    \centering
    \subfloat[Initial dataset]{\includegraphics[width=0.45\linewidth]{images/time_series_analysis/timeseries_dataset.jpg}}
    \subfloat[The organized dataset]{\includegraphics[width=0.55\linewidth]{images/time_series_analysis/timeseries_dataset_modified.jpg}}
    \caption{Time series dataset before and after using the pivot method from pandas.}
    \label{fig:timeseries_dataset}
\end{figure}

We used the k-means clustering from the tslearn package find thhose similar groups of cities, in order to choose the best values of k (the number of clusters) we based our choice, as seen during the clustering task, on the SSE, silhouette and Davies-Bouldin scores. Afterwards, we took the best k value and we used it to do the final clustering on the cities data frame.
\begin{figure}[H]
    \centering
    \subfloat[Scores]{\includegraphics[width=0.38\linewidth]{images/time_series_analysis/timeseries_k.png}}
    \subfloat[The plot with k=5]{\includegraphics[width=0.62\linewidth]{images/time_series_analysis/timeseries_plot_cluster.png}}
    \caption{Scores for each value of k and clustering with the best k value.}
    \label{fig:timeseries_clusters}
\end{figure}
As \autoref{fig:timeseries_clusters} on the left shows, by using the elbow rule and by looking at the other two values we took 5 as the best value for k and the final clustering analysis is shown on \autoref{fig:timeseries_clusters} on the right end.\\

The cities are, therefore, separated in five different cluster in accordance to their temperature trends: cluster 0 with warm winter and very hot summer, cluster 1 composed by cold winter and hot summer, cluster 2 having very hot winter and very hot summer, cluster 3 characterized by warm winter and mild summer and lastly cluster 4 which has very cold winter and mild summer.
\iffalse
\begin{itemize}
    \item Cluster 0, warm winter and very hot summer.
    \item Cluster 1, cold winter and hot summer.
    \item Cluster 2, very hot winter and very hot summer.
    \item Cluster 3, warm winter and mild summer.
    \item Cluster 4, very cold winter and mild summer.
\end{itemize}
\fi
We then plotted those temperature trends on a graph in order to compare them between each other, the cities were, instead, plotted on a map according to the cluster they belong to.
\begin{figure}[H]
    \centering
    \subfloat[Cluster trends]{\includegraphics[width=0.41\linewidth]{images/time_series_analysis/timeseries_plot.png}}
    \subfloat[Cluster map]{\includegraphics[width=0.59\linewidth]{images/time_series_analysis/timeseries_map.png}}
    \caption{Temperature trends for each cluster and the map with the cities plotted with respect to the cluster they belong to.}
    \label{fig:timeseries_plot_map}
\end{figure}
The map in \autoref{fig:timeseries_plot_map} shows that cities in similar clusters are somewhat at the same latitude (which is something that we expected). There could have been some outliers due to high altitude, but most of the cities considered are on plain or hills, the few cities with high altitudes (such as Mexico City) are located near a tropic and are, therefore, hot.
\end{document}
